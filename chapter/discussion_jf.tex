\section{Ergebnisdiskussion}

Bei einem qualitativen Vergleich treten verschiedene unvermeidbare Fehler auf. Es wird unter anderem von modellhaften optimalen Bedingungen ausgegangen. Dabei können jedoch externe Faktoren außer Acht gelassen werden. Dieser Ansatz birgt dadurch das Risiko, dass eben dieser externe Kontext, beispielsweise geografisch bedingt, dafür sorgt, dass reale Tests erheblich von den Modellannahmen abweichen. Eine qualitative Untersuchung des Modells kann daher zu erheblichen Diskrepanzen zwischen der Modellvorstellung und den realen Testergebnissen führen, da externe Einflüsse nicht angemessen berücksichtigt werden.

Ein weiterer Aspekt betrifft die Generalisierung und Vereinfachung komplexer Prozesse durch das Modell. Hierbei besteht die Gefahr, dass möglicherweise unbekannte, aber entscheidende Faktoren bei der Modellbildung übersehen oder vernachlässigt werden. Diese Vereinfachung führt zu einer Verfälschung der Ergebnisse, da wichtige Elemente des komplexen Systems unberücksichtigt bleiben\cite{DisadvantagesOfQualitativApproaches}.

Darüber hinaus muss für einen Vergleich die zu betrachtenden Objekte, hier anonyme Kommunikationssysteme begrenzt nach vielen Kriterien untersuchen werden, um begründet ein Urteil zu fällen. Das Begrenzen der Kriterien wird zwangsläufig jedoch zu einer Unvollständigkeit in der Bewertung führen. Die gewählten Kriterien decken möglicherweise nicht alle relevanten Aspekte ab, und die Fokussierung auf nur wenige Kriterien können zu einem sogenannten \textit{False Balancing} führen. Dies bedeutet, dass ein Modell eine begrenzte Anzahl an Kriterien scheinbar gleich gewichtet, obwohl in der abgebildeten Realität einige Kriterien deutlich schwerer gewichtet sein sollten als andere.

\section{Fazit und Ausblick}

VPNs sind für das Streaming und Messaging geeignet, weil sie eine hohe Leistung bieten und die fehlende Anonymität vernachlässigbar ist.
Mixnets eignen sich für das Banking und Messaging, da hier eine hohe Anonymität geboten wird und der Mangel an Leistung vernachlässigbar ist.
Tor eignet sich aufgrund seiner Ausgewogenheit aus Performance und Anonymität für alle drei Anwendungsbereiche.
Die Ergebnisse dieser Arbeit zeigen, dass der Bereich, für den jemand Tor oder einer der Alternativen nutzen möchte, somit vom individuellen Kontext wie Internetanbieter, Verwendungszweck oder dem Land abhängt. Kein untersuchtes System besitzt sowohl die höchste Anonymität als auch Performance. Daher haben alle Systeme für ihren jeweiligen Anwendungsbereich ihre Daseinsberechtigung.

Diese Arbeit fokussiert sich ausschließlich auf eine qualitative Untersuchung der anonymen Kommunikationssysteme, die Aufzeigt, dass verschiedene Systeme in einem Kriterium besser sind als ein anderes. Was in dieser Arbeit jedoch fehlt, ist eine quantitative Untersuchung, die aufzeigt, um wie viel ein System besser ist als ein anderes. Besonders zur Untersuchung der Performance bieten sich Tests oder Simulationsumgebungen an.
Hinzu kommt, dass bisher nur Systeme anhand von Kriterien bewertet wurden. Es besteht weiterer Forschungsbedarf neu konkrete Lösungsansätze zu entwickeln, um die Kriterien für die Systeme verbessern zu können.