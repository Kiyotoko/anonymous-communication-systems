\section{Tor}
\label{chap:tor}

\subsection{Topologie}
\label{chap:tor_topology}

\begin{figure}[h!]
    \centering
    \includesvg[width=\linewidth]{graph/tor.svg}
    \caption{Tor Topologie}
    \label{imgs:tor}
\end{figure}

Tor ist ein auf den Schutz der Privatsphäre ausgerichtetes Netzwerk, das die Anonymität im Internet verbessern soll, indem es den Internetverkehr über eine Reihe von Servern leitet, die von Freiwilligen betrieben werden und als Onion Router (ORs) bezeichnet werden.

Tor unterteilt seine Knoten in Entry-, Middle- und Exit-Nodes, die jeweils ihre eigenen Funktionen haben.

\begin{itemize}
    \item Entry Nodes: Wenn sich ein Nutzer mit dem Tor-Netzwerk verbindet, wird sein Datenverkehr über einen Eingangsknoten geleitet, der der erste Kontaktpunkt mit dem Tor-Netzwerk ist. Dieser Knoten ist dafür verantwortlich, die äußerste Schicht der Verschlüsselung zu entfernen, wie das Schälen einer Zwiebel, um das nächste Ziel im \glqq Circuit\grqq\ zu enthüllen. Der Einstiegsknoten kennt die IP-Adresse des Benutzers und weiß, woher der Verkehr kommt. Aufgrund der Verschlüsselungsschichten kennt der Eingangsknoten jedoch nicht das endgültige Ziel der Daten.
    \item Middle Nodes: Der mittlere Knoten verbindet den Entry-Node und Exit-Node.
    \item Exit Nodes: Der letzte Knoten in der Schaltung ist der Ausgangsknoten. Der Ausgangsknoten ist dafür verantwortlich, die Anfrage des Benutzers an das gewünschte Ziel im Internet zu senden. Zu diesem Zeitpunkt sind die Daten entschlüsselt worden und liegen in ihrer ursprünglichen Form vor. Daher kennt der Ausgangsknoten das endgültige Ziel des Datenverkehrs, hat aber keine Informationen über die IP-Adresse des Benutzers oder die ursprüngliche Quelle.
\end{itemize}

Tor nutzt Circuit Construction. Um die Anonymität zu wahren, baut Tor \glqq Circuits\grqq\ auf, indem es mehrere Knoten aneinander kettet. Dabei wird immer ein Entry-Node, Middle-Node und Exit-Node miteinander verbunden. Jeder Knoten weiß nur über den vorherigen und den nächsten Knoten im Circuit bescheid, um sicherzustellen, dass keine einzelne Instanz einen vollständigen Überblick über die Online-Aktivitäten des Benutzers hat. Der Eingangsknoten kennt daher nur die IP-Adresse des Benutzers und den nächsten Knoten im Circuit, nicht aber das endgültige Ziel. Die Strecke der Router bleibt dabei somit während des Datenverkehrs gleich, was den Hauptunterschied zu Mixnet darstellt.

\begin{figure}[!h]
    \begin{displaymath}
        m = \mathrm{enc}_{1}\Big(\mathrm{enc}_{3}\big(\mathrm{enc}_{5}(m_0)\big)\Big)
    \end{displaymath}
    \caption{Nachricht wird für den Pfad von Abbildung \ref{imgs:tor} verschlüsselt}
    \label{equa:encryption}
\end{figure}

Hinzu kommt, das Tor für die Verschlüsselung Onion Encryption nutzt. Dabei wird eine Nachricht mit dem öffnetlichen Schlüssel des Exit Nodes verschlüsselt, dann mit dem öffnetlichen Schlüssel des Middle Nodes und zum Schluss mit dem öffentlichen Schlüssel des Entry Nodes (\ref{equa:encryption}). Dabei kann der Entry-Node nur seinen Teil der Nachricht entschlüsseln und diese dann weiter senden. Ziel dabei ist, dass nur der letzte Knotenpunkt die ursprüngliche Nachricht kennt und weiß, mit welchen Service sich der Nutzer verbinden möchte. Dies bedeutet jedoch auch, dass die Nachricht zwischen jedem Knoten verschlüsselt werden muss.

Darüber hinaus wird bei den Nachrichten Padding genutzt. Beim Padding werden Nachrichten unterschiedlicher Länge auf die gleiche Größe $k$ gebracht, um zu verhindern, dass Nachrichten aufgrund ihrer Länge identifiziert werden. Alle Nachrichten, die kleiner als $k$ sind, werden mit solange mit Bits aufgefüllt, bis sie ihre Länge $k$ entspricht. Nachrichten, die größer sind als $k$ werden in mehrere kleinere Nachrichten aufgeteilt, die dann wieder aufgefüllt und aufgeteilt werden, bis ihre Länge $k$ entspricht.

Ein Tor-Server kann einsehen, von welchem vorherigen Server die Nachricht wann erhalten wurde, wie groß sie ist, und welcher Server als nächstes die Nachricht erhält. Darüber hinaus kann der Entry-Node den Sender und der Exit-Node den Empfänger einsehen kann.

Die mehrschichtige Verschlüsselung und das Multi-Hop-Routing in Tor bieten ein hohes Maß an Privatsphäre und Anonymität. Der Eingangsknoten und der Ausgangsknoten haben nur teilweise Informationen über die Daten und ihren Ursprung oder ihr Ziel. Das macht es für jeden, der den Netzwerkverkehr beobachtet, schwierig, die Quelle und das Ziel der Daten zu bestimmen\footnotemark\footnotetext{\cite{TorWhitePaper}, Tor: The Second-Generation Onion Router}.

\subsection{Vorteile}
\label{chap:tor_advantages}

Tor erlaubt durch Circuit Construction hohe Anonymität und Performance. Zwischen jeden der Knoten werden die Datenpakete verschlüsselt.

Der Erfolg von Angriffen auf das Tor-Netzwerk (Kapitel \ref{chap:tor_disatvantages}) hängt unter anderem auch von der Anzahl an Entry- und Exit-Nodes ab. Mit einer größeren Anzahl an Knoten nimmt die Erfolgchance von Angriffen expotenziel ab. Somit können durch ein hohes Angebot an Knoten die Erfolgchance gegen Null gehen, wodurch Tor sehr sicher trotz verschiedener Angriffe wird.

\subsection{Nachteile}
\label{chap:tor_disatvantages}

\begin{figure}[h!]
    \centering
    \includesvg[width=\linewidth]{graph/tor_transfer.svg}
    \caption{Tor Datentransfer}
    \label{imgs:tor_transfer}
\end{figure}

Tor bietet auf Grund seiner Topologie Möglichkeiten für verschiedene Angriffe. Exemplarisch werden hierbei zwei Angriffsmöglichkeiten vorgestellt:

\begin{itemize}
    \item Beim Fingerprinting-Angriff auf Tor versucht der Angreifer, die Webseiten, die ein Nutzer besucht, anhand der Verkehrsmuster und der Merkmale des Datenflusses zu identifizieren. Dieser Angriff ist aufgrund der Designmerkmale von Tor besonders schwierig. Beim ursprünglichen Fingerprinting-Angriff macht sich der Angreifer die Tatsache zunutze, dass Webseiten aus mehreren Dateien bestehen, von denen jede eine bestimmte Dateigröße hat. Durch die Überwachung des Datenverkehrs und das Zählen der Paketgrößen auf verschiedenen Ports kann der Angreifer einen eindeutigen Fingerabdruck für jede Webseite erstellen, der auf der Menge der Dateigrößen basiert.

    Tor, als Anonymitätssystem, stellt diese Art von Angriffen vor gewisse Herausforderungen. Tor verwendet quantisierte Datenzellen mit einer festen Größe, was es für den Angreifer schwierig macht, die Dateigrößen genau zu bestimmen. Außerdem verwendet Tor Multiplexing, um alle TCP-Streams in einer Verbindung zu kombinieren, was es dem Angreifer weiter erschwert, zwischen den Verbindungen zu unterscheiden.

    Das Bedrohungsmodell für diesen Fingerprinting-Angriff geht davon aus, dass der Angreifer den Zugangsrouter des Benutzers besetzt und alle Datenströme des Benutzers beobachtet. Ziel ist es, zu erraten, auf welche Webseite der Benutzer gerade zugreift. Dieser Angriff wird im Vergleich zu anderen Angriffen als ressourcenschonend angesehen und ist daher leichter durchführbar.

    Um den Angriff auf Tor durchzuführen, analysiert der Angreifer den Datenfluss und identifiziert Sequenzen von Paketen. Kurze Intervalle zwischen Outflow-Paketen deuten auf kleine Dateien oder Protokolltransaktionen hin, während längere Intervalle auf größere Dateiübertragungen hindeuten. Indem er diese Intervalle kategorisiert, erstellt der Angreifer einen Vektor, der die Anzahl der Intervalle mit unterschiedlicher Paketanzahl darstellt. Der Angreifer berechnet dann die Ähnlichkeit zwischen diesem Vektor und vordefinierten Fingerprint-Vektoren für verschiedene Webseiten. Die Webseite, die mit dem Fingerabdruckvektor mit dem höchsten Ähnlichkeitswert assoziiert ist, wird als die Webseite betrachtet, auf die der Benutzer am wahrscheinlichsten zugreift.

    Die Wahl des richtigen Fingerabdruckvektors ist entscheidend. Im Zusammenhang mit Tor geht es nicht nur um Dateigrößen, sondern auch darum, die Netzwerkbedingungen des Benutzers widerzuspiegeln. Der Angreifer kann eine Webseite mehrmals aufrufen, Vektoren aufzeichnen und Ähnlichkeitswerte zwischen ihnen berechnen. Der Vektor mit der höchsten Punktzahl wird als Fingerabdruck ausgewählt und repräsentiert sowohl die Eigenschaften der Webseite als auch die Netzwerkbedingungen des Benutzers\footnote{\cite{FingerprintingOnTorAttack}, Fingerprinting attack on the tor anonymity system}.

    \item Bei Timing Attacks wird die Tor-Clients mittels JavaScript-Kombination und Zeit-Analyse deanymisieren. Der Angreifer richtet zwei übernommene Knoten im Tor-Netzwerk ein, wobei einer als Entry-Node und der andere als Exit-Node fungiert. Zusätzlich wird ein Webserver aufgesetzt, um JavaScript-Verbindungen zu protokollieren. Der Angriff nutzt einen übernommenen Tor-Ausgangsknoten, um den gesamten für Tor-Clients bestimmten HTTP-Verkehr zu manipulieren. Hierbei wird ein unsichtbarer JavaScript-Signalgenerator in Webseiten eingefügt.

    Beim Surfen senden die Webbrowser der Tor-Clients kontinuierlich ein eineindeutiges Signal an den übernommenen Server, indem das eingefügte JavaScript ausgeführt wird. Dieser Prozess bleibt so lange aktiv, wie der Benutzer den kompromittierten Browser-Tab geöffnet hält. Etwa alle zehn Minuten wechselt der Tor-Client den Circuit, bis dieser eventuell einen übernommen Entry-Node auswählt.

    Der Angreifer, der den übernommenen Entry-Nodes kontrolliert, führt eine Verkehrsanalyse durch, indem er die Signale, die durch den Knoten laufen, mit den Signalen vergleicht, die der Server empfängt. Wenn eine Übereinstimmung vorliegt, wird die Identität des Tor-Clients zusammen mit dem entsprechenden Verkehrsverlauf offengelegt, während er den kompromittierten Exit-Node benutzt hat. Der Angriff macht sich die Zeitanalyse zunutze, um die Intervalle zwischen Circuit-Wechseln und Signalübertragungen zu ermitteln.

    Obwohl Entry-Nodes eine Standardfunktion in Tor sind und diesen Angriff teilweise entschärfen, bleibt die Methode dennoch effektiv. Der Angriff kann durch den Einsatz mehrerer übernommener Tor-Entry-Node skaliert werden, was die Zeit zur Kompromittierung der Anonymität eines Clients verringert. Die Verwendung eines Signalgenerators ermöglicht die Entkopplung der Kontrolle über Entry- und Exit-Nodes, was den Prozess weiter vereinfacht.

    JavaScript stellt hierbei exemplarisch nur eine Methode dar, Signale auszusenden und somit Daten zum deanymisieren zu sammeln. Andere Sprachen und Methoden sind ebenfalls möglich\footnote{\cite{BrowserBasedAttacksOnTor}, Browser-Based Attacks on Tor}.
\end{itemize}
