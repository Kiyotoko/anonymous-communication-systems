\section{Tor}

\subsection{Topologie}

\begin{figure}[h!]
    \centering
    \includesvg[width=\linewidth]{graph/tor.svg}
    \caption{Tor}
    \label{imgs:tor}
\end{figure}

Tor, die Abkürzung für \glqq The Onion Router\grqq, ist ein auf den Schutz der Privatsphäre ausgerichtetes Netzwerk, das die Anonymität im Internet verbessern soll, indem es den Internetverkehr über eine Reihe von Servern leitet, die von Freiwilligen betrieben werden und als Onion Router (ORs) bezeichnet werden. Dieses Netzwerk arbeitet als Overlay über der bestehenden Internet-Infrastruktur. Um die Funktionsweise von Tor besser erläutern, werden im nachfolgenden die wichtigsten Bestandteile und ihre Funktion dargelegt:

\begin{itemize}

\item Entry Nodes: Wenn sich ein Nutzer mit dem Tor-Netzwerk verbindet, wird sein Datenverkehr über einen Eingangsknoten geleitet, der der erste Kontaktpunkt mit dem Tor-Netzwerk ist. Dieser Knoten ist dafür verantwortlich, die äußerste Schicht der Verschlüsselung zu entfernen, wie das Schälen einer Zwiebel, um das nächste Ziel im Kreislauf zu enthüllen. Der Einstiegsknoten kennt die IP-Adresse des Benutzers und weiß, woher der Verkehr kommt. Aufgrund der Verschlüsselungsschichten kennt der Eingangsknoten jedoch nicht das endgültige Ziel der Daten.

\item Circuit Construction: Um die Anonymität zu wahren, baut Tor \glqq circuits\grqq\ auf, indem es mehrere Onion-Router aneinander kettet. Jeder Router weiß nur über den vorherigen und den nächsten Router im Kreislauf Bescheid, um sicherzustellen, dass keine einzelne Instanz einen vollständigen Überblick über die Online-Aktivitäten des Benutzers hat. Der Eingangsknoten kennt daher nur die IP-Adresse des Benutzers und den nächsten Router im Kreislauf, nicht aber das endgültige Ziel.

\item Exit Nodes: Der letzte Knoten in der Schaltung ist der Ausgangsknoten. Der Ausgangsknoten ist dafür verantwortlich, die Anfrage des Benutzers an das gewünschte Ziel im Internet zu senden. Zu diesem Zeitpunkt sind die Daten entschlüsselt worden und liegen in ihrer ursprünglichen Form vor. Daher kennt der Ausgangsknoten das endgültige Ziel des Datenverkehrs, hat aber keine Informationen über die IP-Adresse des Benutzers oder die ursprüngliche Quelle.

\end{itemize}

Ein Tor-Server kann einsehen, von welchem vorherigen Server die Nachricht wann erhalten wurde, wie groß sie ist, und welcher Server als nächstes die Nachricht erhält. Darüber hinaus kann der Entry-Node den Sender und der Exit-Node den Empfänger einsehen kann.

Die mehrschichtige Verschlüsselung und das Multi-Hop-Routing in Tor bieten ein hohes Maß an Privatsphäre und Anonymität. Der Eingangsknoten und der Ausgangsknoten haben nur teilweise Informationen über die Daten und ihren Ursprung oder ihr Ziel. Das macht es für jeden, der den Netzwerkverkehr beobachtet, schwierig, die Quelle und das Ziel der Daten zu bestimmen\footnotemark\footnotetext{\cite{TorWhitePaper}, Tor: The Second-Generation Onion Router}.

\subsection{Vorteile}

\subsection{Nachteile}

Beim Fingerprinting-Angriff auf Tor versucht der Angreifer, die Webseiten, die ein Nutzer besucht, anhand der Verkehrsmuster und der Merkmale des Datenflusses zu identifizieren. Dieser Angriff ist aufgrund der Designmerkmale von Tor besonders schwierig. Beim ursprünglichen Fingerprinting-Angriff macht sich der Angreifer die Tatsache zunutze, dass Webseiten aus mehreren Dateien bestehen, von denen jede eine bestimmte Dateigröße hat. Durch die Überwachung des Datenverkehrs und das Zählen der Paketgrößen auf verschiedenen Ports kann der Angreifer einen eindeutigen Fingerabdruck für jede Webseite erstellen, der auf der Menge der Dateigrößen basiert.

Tor, als Anonymitätssystem, stellt diese Art von Angriffen vor gewisse Herausforderungen. Tor verwendet quantisierte Datenzellen mit einer festen Größe, was es für den Angreifer schwierig macht, die Dateigrößen genau zu bestimmen. Außerdem verwendet Tor Multiplexing, um alle TCP-Streams in einer Verbindung zu kombinieren, was es dem Angreifer weiter erschwert, zwischen den Verbindungen zu unterscheiden.

Das Bedrohungsmodell für diesen Fingerprinting-Angriff geht davon aus, dass der Angreifer den Zugangsrouter des Benutzers besetzt und alle Datenströme des Benutzers beobachtet. Ziel ist es, zu erraten, auf welche Webseite der Benutzer gerade zugreift. Dieser Angriff wird im Vergleich zu anderen Angriffen als ressourcenschonend angesehen und ist daher leichter durchführbar.

Um den Angriff auf Tor durchzuführen, analysiert der Angreifer den Datenfluss und identifiziert Sequenzen von Paketen. Kurze Intervalle zwischen Outflow-Paketen deuten auf kleine Dateien oder Protokolltransaktionen hin, während längere Intervalle auf größere Dateiübertragungen hindeuten. Indem er diese Intervalle kategorisiert, erstellt der Angreifer einen Vektor, der die Anzahl der Intervalle mit unterschiedlicher Paketanzahl darstellt. Der Angreifer berechnet dann die Ähnlichkeit zwischen diesem Vektor und vordefinierten Fingerprint-Vektoren für verschiedene Webseiten. Die Webseite, die mit dem Fingerabdruckvektor mit dem höchsten Ähnlichkeitswert assoziiert ist, wird als die Webseite betrachtet, auf die der Benutzer am wahrscheinlichsten zugreift.

Die Wahl des richtigen Fingerabdruckvektors ist entscheidend. Im Zusammenhang mit Tor geht es nicht nur um Dateigrößen, sondern auch darum, die Netzwerkbedingungen des Benutzers widerzuspiegeln. Der Angreifer kann eine Webseite mehrmals aufrufen, Vektoren aufzeichnen und Ähnlichkeitswerte zwischen ihnen berechnen. Der Vektor mit der höchsten Punktzahl wird als Fingerabdruck ausgewählt und repräsentiert sowohl die Eigenschaften der Webseite als auch die Netzwerkbedingungen des Benutzers\footnote{\cite{FingerprintingOnTorAttack}, Fingerprinting attack on the tor anonymity system}.
