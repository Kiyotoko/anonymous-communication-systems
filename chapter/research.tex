\section{Recherche}

\subsection{Erläuterung von Begriffen}

\subsubsection{Overlay Networks}

Ein Overlay Network besteht aus Servern, die über das Internet verteilt sind und als zusätzliche Schicht fungieren, die einer oder mehreren Anwendungen die Infrastruktur zur Verfügung stellt. Entscheidend ist, dass diese Overlays die Verantwortung für die Weiterleitung und Verwaltung von Anwendungsdaten übernehmen und dabei oft von der grundlegenden Internet-Infrastruktur abweichen oder mit ihr konkurrieren.\footnote{\cite{OverlayNetwork}, Overlay Network}

Overlay-Netzwerke ermöglichen eine strukturierte virtuelle Topologie, die über der grundlegenden Transportprotokollebene liegt und eine deterministische Suche erleichtert sowie Konvergenz gewährleistet. Dadurch können sie eine Reihe fortgeschrittener Funktionen anbieten, die über das hinausgehen, was das grundlegende Internet mit seiner physischen Topologie bietet. Sie können Mobilität, individuelles Routing, Quality of Service (QoS), neuartige Adressierung, verbesserte Sicherheit, Multicast und die Verteilung von Inhalten ermöglichen. So können Overlays beispielsweise Anwendungen unterstützen, die Anonymität erfordern, indem sie den Datenverkehr über zwischengeschaltete Server umleiten und so die wahre Quelle und das wahre Ziel der Daten verschleiern.

Diese Overlay Networks stellen auch die traditionellen End-to-End-Konstruktionsprinzipien wie P2P in Frage, die in der Vergangenheit die Platzierung von Funktionen und Diensten innerhalb der Internet-Architektur bestimmt haben\footnote{\cite{AlternativeToP2P}, Overlay Networks: A scalable alternative to P2P}. Indem sie neu definieren, wo und wie diese Funktionalitäten platziert werden, tragen Overlays zur Weiterentwicklung der Internetstruktur bei und ermöglichen die Anpassung an die sich ändernden Anforderungen von Benutzern und Anwendungen\footnote{\cite{FutureOfTheInternet}, Overlay Networks and the Future of the Internet}.

\subsubsection{Mixnet}

\begin{figure}[h!]
    \centering
    \includesvg[width=\linewidth]{graph/mixnet.svg}
    \caption{Mixnet}
    \label{imgs:mixnet}
\end{figure}

Ein Mixnet ist ein System, das aus mehreren Servern besteht, die als MIXe oder MIX-Knoten bezeichnet werden. Jeder dieser MIXe ist mit einem öffentlichen Schlüssel verbunden. Der Hauptzweck eines MIX-Netzes besteht darin, eine Möglichkeit zu bieten, Nachrichten anonym und privat über ein Netzwerk zu versenden.

Ein MIX-Netz funktioniert wie folgt:

\begin{itemize}

\item Verschlüsselung von Nachrichten: Die Benutzer senden verschlüsselte Nachrichten an das MIX-Netz.

\item Verarbeitung der Nachricht: Nach dem Empfang der verschlüsselten Nachrichten entschlüsselt jeder MIX die empfangenen Nachrichten mit seinem privaten Schlüssel. Dies ermöglicht es dem MIX, die Nachrichten zu verarbeiten.

\item Stapelung und Permutation: Die entschlüsselten Nachrichten werden dann in Stapeln gruppiert. Die Reihenfolge dieser Nachrichten wird permutiert, d. h. sie werden in einer zufälligen Reihenfolge gemischt. Diese Vermischung der Nachrichtenreihenfolge trägt dazu bei, die Verbindung zwischen dem Absender und dem Empfänger zu unterbrechen.

\item Entfernen der Identität des Absenders: Vor der Weiterleitung der Nachrichten entfernt jeder MIX alle Informationen, die den Absender identifizieren könnten, wie z. B. den Namen des Absenders und andere identifizierende Details. Dieser Schritt erhöht die Anonymität weiter.

\item Weiterleitung: Nachdem die Identität des Absenders entfernt und die Reihenfolge der Nachrichten vertauscht wurde, leitet der MIX die verarbeiteten Nachrichten an den nächsten MIX in der Reihe weiter\footnote{\cite{ComposableMixNet}, A Universally Composable Mix-Net}.

\end{itemize}

Die Bedeutung dieses Prozesses liegt in seiner Fähigkeit, den Kommunikationsteilnehmern Anonymität zu bieten. Ein Beobachter, selbst wenn er die gesamte Kommunikation zwischen MIXen belauschen kann, würde es aufgrund der durch den Permutationsprozess eingeführten Randomisierung äußerst schwierig finden, bestimmte Nachrichten mit ihren ursprünglichen Absendern und Empfängern zu verknüpfen.

Die Sicherheit von MIXen ist von grundlegender Bedeutung, dass selbst passive Angreifer (die nur die Kommunikation abhören) die vom MIX-Netz gebotene Anonymität nicht ohne weiteres brechen können. Die Sicherheitseigenschaft garantiert, dass ein Lauscher zwar einige Informationen aus der Beobachtung der Nachrichten zwischen MIXen gewinnen kann, dieser Vorteil aber vernachlässigbar ist und keine genaue Erkennung der Sender-Empfänger-Beziehungen ermöglicht.

MIX-Netze haben sich über das ursprüngliche Konzept hinaus weiterentwickelt und zu verschiedenen Forschungsrichtungen und Verbesserungen geführt:

\begin{itemize}

\item Stop-and-Go-MIX-Netze: Hierbei handelt es sich um Varianten von MIX-Netzen, die kontrollierte Pausen in den Nachrichtenverarbeitungsfluss einführen, um die Sicherheit zu erhöhen.

\item Verteilte \glqq Flash-MIXe\grqq: Hier wird der Mischprozess auf mehrere Server verteilt, um die Effizienz und Anonymität zu verbessern. Es wurden jedoch auch Schwachstellen in diesen Systemen festgestellt.

\item Hybride MIXe: In dieser Richtung werden verschiedene Techniken kombiniert, um eine verbesserte Privatsphäre und Sicherheit zu erreichen\footnotemark\footnotetext{\cite{MIXNetReliability}, A Reputation System to Increase MIX-Net Reliability}.

\end{itemize}.

Ein Knoten kann sehen, von welchem Server ein Paket gesendet wurde, und an welchem Server es weitergeleitet wird. Abhängig vom MIX-Netz kann ein Knoten jedoch weder den Zeitpunkt noch die Größe des ursprüngliche Paketes bestimmen. Auch kann ein Knoten nicht feststellen, ob ein Nutzer überhaupt kommuniziert.

\subsubsection{Tor}

\begin{figure}[h!]
    \centering
    \includesvg[width=\linewidth]{graph/tor.svg}
    \caption{Tor}
    \label{imgs:tor}
\end{figure}

Tor, die Abkürzung für \glqq The Onion Router\grqq, ist ein auf den Schutz der Privatsphäre ausgerichtetes Netzwerk, das die Anonymität im Internet verbessern soll, indem es den Internetverkehr über eine Reihe von Servern leitet, die von Freiwilligen betrieben werden und als Onion Router (ORs) bezeichnet werden. Dieses Netzwerk arbeitet als Overlay über der bestehenden Internet-Infrastruktur. Um die Funktionsweise von Tor besser erläutern, werden im nachfolgenden die wichtigsten Bestandteile und ihre Funktion dargelegt:

\begin{itemize}

\item Entry Nodes: Wenn sich ein Nutzer mit dem Tor-Netzwerk verbindet, wird sein Datenverkehr über einen Eingangsknoten geleitet, der der erste Kontaktpunkt mit dem Tor-Netzwerk ist. Dieser Knoten ist dafür verantwortlich, die äußerste Schicht der Verschlüsselung zu entfernen, wie das Schälen einer Zwiebel, um das nächste Ziel im Kreislauf zu enthüllen. Der Einstiegsknoten kennt die IP-Adresse des Benutzers und weiß, woher der Verkehr kommt. Aufgrund der Verschlüsselungsschichten kennt der Eingangsknoten jedoch nicht das endgültige Ziel der Daten.

\item Circuit Construction: Um die Anonymität zu wahren, baut Tor \glqq circuits\grqq\ auf, indem es mehrere Onion-Router aneinander kettet. Jeder Router weiß nur über den vorherigen und den nächsten Router im Kreislauf Bescheid, um sicherzustellen, dass keine einzelne Instanz einen vollständigen Überblick über die Online-Aktivitäten des Benutzers hat. Der Eingangsknoten kennt daher nur die IP-Adresse des Benutzers und den nächsten Router im Kreislauf, nicht aber das endgültige Ziel.

\item Exit Nodes: Der letzte Knoten in der Schaltung ist der Ausgangsknoten. Der Ausgangsknoten ist dafür verantwortlich, die Anfrage des Benutzers an das gewünschte Ziel im Internet zu senden. Zu diesem Zeitpunkt sind die Daten entschlüsselt worden und liegen in ihrer ursprünglichen Form vor. Daher kennt der Ausgangsknoten das endgültige Ziel des Datenverkehrs, hat aber keine Informationen über die IP-Adresse des Benutzers oder die ursprüngliche Quelle.

\end{itemize}

Ein Tor-Server kann einsehen, von welchem vorherigen Server die Nachricht wann erhalten wurde, wie groß sie ist, und welcher Server als nächstes die Nachricht erhält. Darüber hinaus kann der Entry-Node den Sender und der Exit-Node den Empfänger einsehen kann.

Die mehrschichtige Verschlüsselung und das Multi-Hop-Routing in Tor bieten ein hohes Maß an Privatsphäre und Anonymität. Der Eingangsknoten und der Ausgangsknoten haben nur teilweise Informationen über die Daten und ihren Ursprung oder ihr Ziel. Das macht es für jeden, der den Netzwerkverkehr beobachtet, schwierig, die Quelle und das Ziel der Daten zu bestimmen\footnotemark\footnotetext{\cite{TorWhitePaper}, Tor: The Second-Generation Onion Router}.

\subsubsection{VPN}

\begin{figure}[h!]
    \centering
    \includesvg{graph/vpn.svg}
    \caption{VPN}
    \label{imgs:vpn}
\end{figure}

Ein virtuelles privates Netzwerk (VPN) ist eine Kommunikationsumgebung, die eine Kombination aus Privatsphäre und Virtualisierung innerhalb einer gemeinsamen Netzwerkinfrastruktur bietet. Um dieses Konzept zu verstehen, wird nun die Erklärung im Text aufgeschlüsselt:

\begin{itemize}

\item Netzwerk: Ein Netzwerk besteht aus Geräten (wie Computern, Druckern, Routern), die mit verschiedenen Methoden kommunizieren können. Diese Kommunikation kann über verschiedene Standorte hinweg erfolgen und wird durch elektronische Signalisierungsspezifikationen und -protokolle ermöglicht.

\item Privat: Im Zusammenhang mit VPNs bedeutet \glqq privat\grqq, dass die Kommunikation zwischen Geräten vertraulich ist. Diejenigen, die nicht an diesem privaten Austausch teilnehmen, haben keine Kenntnis von dessen Inhalt und der privaten Verbindung selbst. Die Gewährleistung des Datenschutzes und der Datensicherheit ist bei der Implementierung von VPNs von entscheidender Bedeutung.

\item Virtuell: \glqq Virtuell\grqq\ bezieht sich auf simulierte oder künstliche Aspekte, die von Computersystemen geschaffen werden, um gemeinsame Ressourcen zu verwalten. Bei VPNs beinhaltet die Virtualisierung die private Kommunikation über eine gemeinsam genutzte Netzinfrastruktur. Das private Netz wird durch logische Partitionierung der gemeinsamen Ressource aufgebaut, nicht durch dedizierte physische Schaltungen.

\item Diskret: VPNs sind diskrete Netzwerke, die getrennt über eine gemeinsam genutzte Infrastruktur arbeiten. Sie bieten exklusive Kommunikationsumgebungen ohne gemeinsame Verbindungspunkte.

\end{itemize}

Kombiniert man diese Aspekte, ist ein VPN ein privates Netzwerk, das durch Virtualisierungsmethoden eingeführt wird. Es handelt sich um eine kontrollierte Kommunikationsumgebung, in der der Zugang nur einer bestimmten Interessengemeinschaft gestattet ist. Diese Umgebung wird durch die Partitionierung eines gemeinsamen Kommunikationsmediums gebildet. Während der Begriff \glqq privates Netzwerk\grqq\ irreführend sein könnte, da alle Netzwerke bis zu einem gewissen Grad auf einer gemeinsamen Infrastruktur beruhen, unterscheiden sich VPNs durch die Segmentierung der Kommunikation innerhalb einer gemeinsamen Infrastruktur.

Ein VPN-Server kann einsehen, was wann von welchem Nutzer an welchen Server gesendet wird. Ein solcher Server wäre in der Lage, sämtliche relavanten Informationen zu erfassen und zu speichern.

In einem formalen Sinne wird ein VPN als eine Kommunikationsumgebung definiert, die den Zugang kontrolliert, um Peer-Verbindungen innerhalb einer bestimmten Interessengemeinschaft zu erleichtern. Diese Umgebung basiert auf der Partitionierung eines gemeinsam genutzten Kommunikationsmediums, das nicht-exklusive Dienste für das Netz bereitstellt\footnote{\cite{DefinitionOfVPN}, What is a VPN?}.

\subsection{Anonyme Kommunikationssysteme}

\subsubsection{Vorteile}

Je nach den spezifischen strukturellen Merkmalen können anonyme Kommunikationssysteme eine Vielzahl von Vorteilen bereithalten, die wiederum abhängig von den individuellen Anforderungen und Einsatzszenarien zur vollen Entfaltung gelangen können. In dieser Arbeit werden vor allem die folgenden Vorteile betrachtet:

\begin{itemize}

\item Erhöhte Anonymität: Anonyme Kommunikationssysteme bieten den Nutzern ein gesteigertes Maß an Anonymität und Datenschutz. Diese Systeme erschweren die Identifikation von Einzelpersonen und schützen so deren persönliche Daten vor neugierigen Blicken. Durch die Verschleierung der Identität wird die Privatsphäre gestärkt, was besonders in Zeiten erhöhter Online-Überwachung und Datenverfolgung von entscheidender Bedeutung ist.

\item Internetzensur umgehen: Die Flexibilität von anonymen Kommunikationssystemen ermöglicht es den Nutzern, Internetzensur zu umgehen und auf Informationen zuzugreifen, die sonst blockiert oder zensiert sind. In Ländern mit strenger Kontrolle des Informationsflusses können solche Systeme eine wertvolle Rolle spielen, indem sie den Zugang zu unzensierten Inhalten, kritischen Nachrichten und alternativen Standpunkten gewährleisten.

\item Geoblocking: Anonyme Kommunikationssysteme ermöglichen es den Nutzern, geografische Beschränkungen zu überwinden, die oft durch Geoblocking-Maßnahmen auferlegt werden. Indem sie ihren Standort verschleiern, können Nutzer auf Inhalte und Dienste zugreifen, die normalerweise auf bestimmte Regionen beschränkt sind. Dies ist besonders nützlich für Menschen, die auf reisespezifische oder länderspezifische Inhalte zugreifen möchten, unabhängig von ihrem Aufenthaltsort.

\end{itemize}

\subsubsection{Nachteile}

Die Geschwindigkeit, mit der Daten über Anonyme Kommunikationssysteme abgerufen werden können, kann abhängig vom System teils deutlich langsamer sein also über die Anfragen. Dies hängt unter anderem von der Verschlüsselung ab. Verschiede Verschlüsselungsverfahren laufen auch unterschiedlich schnell, was jedoch auf Kosten der Sicherheit kommt. Umso häufiger eine Nachricht Verschlüsselt werden muss, umso größer ist auch die Verzögerung, die Entsteht\footnote{\cite{EffectivenessOfMixnets}}.

\subsubsection{Schwachstellen}

Beim Fingerprinting-Angriff auf Tor versucht der Angreifer, die Webseiten, die ein Nutzer besucht, anhand der Verkehrsmuster und der Merkmale des Datenflusses zu identifizieren. Dieser Angriff ist aufgrund der Designmerkmale von Tor besonders schwierig. Beim ursprünglichen Fingerprinting-Angriff macht sich der Angreifer die Tatsache zunutze, dass Webseiten aus mehreren Dateien bestehen, von denen jede eine bestimmte Dateigröße hat. Durch die Überwachung des Datenverkehrs und das Zählen der Paketgrößen auf verschiedenen Ports kann der Angreifer einen eindeutigen Fingerabdruck für jede Webseite erstellen, der auf der Menge der Dateigrößen basiert.

Tor, als Anonymitätssystem, stellt diese Art von Angriffen vor gewisse Herausforderungen. Tor verwendet quantisierte Datenzellen mit einer festen Größe, was es für den Angreifer schwierig macht, die Dateigrößen genau zu bestimmen. Außerdem verwendet Tor Multiplexing, um alle TCP-Streams in einer Verbindung zu kombinieren, was es dem Angreifer weiter erschwert, zwischen den Verbindungen zu unterscheiden.

Das Bedrohungsmodell für diesen Fingerprinting-Angriff geht davon aus, dass der Angreifer den Zugangsrouter des Benutzers besetzt und alle Datenströme des Benutzers beobachtet. Ziel ist es, zu erraten, auf welche Webseite der Benutzer gerade zugreift. Dieser Angriff wird im Vergleich zu anderen Angriffen als ressourcenschonend angesehen und ist daher leichter durchführbar.

Um den Angriff auf Tor durchzuführen, analysiert der Angreifer den Datenfluss und identifiziert Sequenzen von Paketen. Kurze Intervalle zwischen Outflow-Paketen deuten auf kleine Dateien oder Protokolltransaktionen hin, während längere Intervalle auf größere Dateiübertragungen hindeuten. Indem er diese Intervalle kategorisiert, erstellt der Angreifer einen Vektor, der die Anzahl der Intervalle mit unterschiedlicher Paketanzahl darstellt. Der Angreifer berechnet dann die Ähnlichkeit zwischen diesem Vektor und vordefinierten Fingerprint-Vektoren für verschiedene Webseiten. Die Webseite, die mit dem Fingerabdruckvektor mit dem höchsten Ähnlichkeitswert assoziiert ist, wird als die Webseite betrachtet, auf die der Benutzer am wahrscheinlichsten zugreift.

Die Wahl des richtigen Fingerabdruckvektors ist entscheidend. Im Zusammenhang mit Tor geht es nicht nur um Dateigrößen, sondern auch darum, die Netzwerkbedingungen des Benutzers widerzuspiegeln. Der Angreifer kann eine Webseite mehrmals aufrufen, Vektoren aufzeichnen und Ähnlichkeitswerte zwischen ihnen berechnen. Der Vektor mit der höchsten Punktzahl wird als Fingerabdruck ausgewählt und repräsentiert sowohl die Eigenschaften der Webseite als auch die Netzwerkbedingungen des Benutzers\footnote{\cite{FingerprintingOnTorAttack}, Fingerprinting attack on the tor anonymity system}.