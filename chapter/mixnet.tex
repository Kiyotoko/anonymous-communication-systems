\section{Mixnet}
\label{chap:mixnet}

\subsection{Topologie}
\label{chap:mixnet_topology}

Ein Mixnet ist ein System, das aus mehreren Servern besteht, die als MIXe oder MIX-Knoten bezeichnet werden. Jeder dieser MIXe ist mit einem öffentlichen Schlüssel verbunden. Der Hauptzweck eines MIX-Netzes besteht darin, eine Möglichkeit zu bieten, Nachrichten anonym und privat über ein Netzwerk zu versenden.

Ein MIX-Netz funktioniert wie folgt:

\begin{itemize}
    \item Verschlüsselung von Nachrichten: Die Nutzer wählen eine zufällige Route durch das MIX-Netz. Auf die Nachricht wird Padding angewendet (Kapitel \ref{chap:tor_topology}), dh. die Nachricht wird so aufgefüllt, dass sie gleich lang wie andere ist. Mithilfe von Onion Encryption wird die Nachricht verschlüsselt und zum ersten MIX-Knoten versendet.
    \item Verarbeitung der Nachricht: Nach dem Empfang der verschlüsselten Nachrichten entschlüsselt jeder MIX die empfangenen Nachrichten mit seinem privaten Schlüssel. Dies ermöglicht es dem MIX, die Nachrichten zu verarbeiten.
    \item Stapelung und Permutation: Die entschlüsselten Nachrichten werden dann in Stapeln gruppiert. Die Reihenfolge dieser Nachrichten wird permutiert, d. h. sie werden in einer zufälligen Reihenfolge gemischt. Diese Vermischung der Nachrichtenreihenfolge trägt dazu bei, die Verbindung zwischen dem Absender und dem Empfänger zu unterbrechen.
    \item Entfernen der Identität des Absenders: Vor der Weiterleitung der Nachrichten entfernt jeder MIX alle Informationen, die den Absender identifizieren könnten, wie z. B. den Namen des Absenders und andere identifizierende Details. Dieser Schritt erhöht die Anonymität weiter.
    \item Weiterleitung: Nachdem die Identität des Absenders entfernt und die Reihenfolge der Nachrichten vertauscht wurde, leitet der MIX die verarbeiteten Nachrichten an den nächsten MIX in der Reihe weiter\footnote{\cite{ComposableMixNet}, A Universally Composable Mix-Net}.
\end{itemize}

\begin{figure}[h!]
    \centering
    \includesvg[width=\linewidth]{graph/mixnet.svg}
    \caption{Die Nutzer wählen jedes mal zufällig eine neu Route durch das Mixnet zum Service. Dabei hat jeder Knoten die selbe wahrscheinlichkeit von allen Nutzern gewählt zu werden wie jeder andere.}
    \label{imgs:mixnet}
\end{figure}

Beim Routing von Mixnet wird anders als bei Tor keine Circuits aufgebaut, sondern es wird jedes mal eine neue Route zufällig gewählt. Dabei hat jeder Knoten innerhalb der selben Ebene die gleiche wahrscheinlichkeit bei allen Nutzern ausgewählt zu werden wie jeder andere (Abbildung \ref{imgs:mixnet}). Wie auch bei Tor unterteilt auch Mixnet die Knoten in Entry-, Middle- und Exit-Nodes.

\begin{figure}[h!]
    \centering
    \includesvg[width=\linewidth]{graph/mixnet_stack.svg}
    \caption{Mixnet ändert in einem Stapel die Permutation, dh. dass die Reihenfolge der eingehenden Nachrichten nicht der Reihenfolge der ausgehenden Nachrichten entspricht.}
    \label{imgs:mixnet_stack}
\end{figure}

Bei Mixnet-Knoten wird ein Stack verwendet (Abbildung \ref{imgs:mixnet_stack}). Dieser funktioniert, indem die MIX die Nachrichten sammeln. Nachdem sie eine gewisse Anzahl an Nachrichten erhalten haben, werden alle erhaltenen Nachrichten zufällig neu gemischt (Mixing). Die Nachrichten werden dann in der neu gemischten Reihenfolge wieder versendet. Ziel dabei ist, dass die Reihenfolge der gesendeten Nachrichten nicht der Reihenfolge der erhaltenen Nachrichten entspricht, und somit durch das passive Betrachten der Kommunikation bei einem MIX keine Rückschlüsse auf den Ursprung der versendeten Nachrichten ziehen kann. Jedoch entsteht dadurch, dass der MIX zunächst die Nachrichten sammeln muss, eine Verzögerung.

Ein Knoten kann sehen, von welchem Server ein Paket gesendet wurde, und an welchem Server es weitergeleitet wird. Ein Knoten jedoch kann weder den Zeitpunkt noch die Größe des ursprüngliche Paketes bestimmen, da Padding auf die Nachrichten angewendet wird . Auch kann ein Knoten nicht feststellen, ob ein Nutzer überhaupt kommuniziert.

Die Bedeutung dieses Prozesses liegt in seiner Fähigkeit, den Kommunikationsteilnehmern Anonymität zu bieten, indem Pfade beim Routing zufällig ausgewählt werden, Padding und Mixing angewendet wird, sowie Nachrichten durch Onion Encryption verschlüsselt sind. Ein Beobachter, selbst wenn er die gesamte Kommunikation zwischen MIXen verfolgt, könnte nicht Nachrichten mit ihren ursprünglichen Absendern und Empfängern verknüpfen. Somit sind zwei Nachrichten unterschiedlicher Nutzer nicht voneinander im Netzwerk unterscheidbar (Abbildung \ref{imgs:mixnet_transfer}).

\begin{figure}[h!]
    \centering
    \includesvg[width=\linewidth]{graph/mixnet_transfer.svg}
    \caption{Bei Mixnet sind zwei Nutzer nicht voneinander unterscheidbar.}
    \label{imgs:mixnet_transfer}
\end{figure}

Die Sicherheit von MIXen ist von grundlegender Bedeutung, da Angreifer die vom MIX-Netz gebotene Anonymität nicht ohne weiteres brechen können. Die Sicherheitseigenschaft garantiert, dass ein Dritte zwar Informationen aus der Beobachtung der Nachrichten zwischen MIXen gewinnen könnte, dieser Vorteil aber vernachlässigbar ist und keine genaue Erkennung der Sender-Empfänger-Beziehungen ermöglicht.

\subsection{Weiterentwicklungen}
\label{chap:mixnet_enhancements}

MIX-Netze haben sich über das ursprüngliche Konzept hinaus weiterentwickelt und zu verschiedenen Forschungsrichtungen und Verbesserungen geführt:

\begin{itemize}
    \item Stop-and-Go-MIX-Netze: Hierbei handelt es sich um Varianten von MIX-Netzen, die kontrollierte Pausen in den Nachrichtenverarbeitungsfluss einführen, um die Sicherheit zu erhöhen.
    \item Verteilte \glqq Flash-MIXe\grqq: Hier wird der Mischprozess auf mehrere Server verteilt, um die Effizienz und Anonymität zu verbessern. Es wurden jedoch auch Schwachstellen in diesen Systemen festgestellt.
    \item Hybride MIXe: In dieser Richtung werden verschiedene Techniken kombiniert, um eine verbesserte Privatsphäre und Sicherheit zu erreichen\footnote{\cite{MIXNetReliability}, A Reputation System to Increase MIX-Net Reliability}.
\end{itemize}.

\subsection{Vorteile}
\label{chap:mixnet_advantages}

\begin{itemize}
    \item Timig Attacks sind im Gegensatz zu anderen Systemen nicht möglich, da die die Reihenfolge der Nachrichten, indem diese ankommen, zufällig verändert und verzögert wird (Kapitel \ref{chap:tor_disatvantages}). Dadurch entpsricht die Reihenfolge der abgesendeten Nachrichten nicht der Reihenfolge der Nachrichten, die angekommen sind, wodurch diese Informationen gegenüber Dritten unbekannt ist. Da alle Knoten dies Anwenden, können äußere als auch einzelne Knoten selbst nicht die Nachricht nachverfolgen\footnote{\cite{MixNetworksSecureApplications}, A Survey on Mix Networks and Their Secure Applications}.
    \item Es kann bei Systemen die zusätzlich Fake Traffik senden, nicht festgestellt werden, ob überhaupt kommuniziert wird. Fake Traffik wird von Nutzern zusätzlich zu echten Nachrichten im Netzwerk gesendet. Wenn eine Nachricht von einem Nutzer ausgesendet wird, 
    \item Mixnets können Abhänging von der Implementierung gut monetarisierbar sein. Nym stellt dafür ein System vor, wo auf dem Mixnet eine Blockchain für Cryptowährungen gebaut sind. Dabei werden gewinne aus der Blockchain mit den Server Hostern geteilt. Dies ermöglicht es, finanziell die Kosten der Hoster zu decken und somit eine langfristig nachhaltige Finanzierung zu gewährleisten\footnote{\cite{RewardSharingForMixnets}, Reward Sharing for Mixnets}.
\end{itemize}

\subsection{Nachteile}
\label{chap:mixnet_disadvantages}

\begin{itemize}
    \item Latenz ist bei Mixnets deutlich höher. Dies hat zwei Gründe. Primär liegt es daran, dass bei Stop-and-Go-MIX-Netze die Verzögerung die Latenz deutlich erhöht, da immer im Stack gewartet wird, bis genug Nachrichten angekommen sind, um diese dann anschließend zu Mixen. Hinzu kommt, dass zwischen jedem Knoten die Daten verschlüsselt werden mpssen. Umso häufiger eine Nachricht Verschlüsselt wird, umso größer ist auch die Verzögerung, welche entsteht\footnote{\cite{EffectivenessOfMixnets}, The effectiveness of mixnets - an empirical study}.
    \item Der reale Durchsatz ist minimal geringer als vergleichbare Systeme. Als realer Durchsatz wird der Durchsatz an echten Nachrichten bezeichnet, Fake Traffik zählt somit nicht darunter. Jedoch kann Fake Traffik den Nachteil der Latenz minimieren, da dadurch mehr Nachrichten versendet werden, wodurch der Stack kürzer auf andere Nachrichten warten muss.
\end{itemize}
