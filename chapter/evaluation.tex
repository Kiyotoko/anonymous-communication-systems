\section{Auswertung}

\subsection{Ergebnisse}

\subsubsection{Anonymität}

Mixnet bietet die höchste Stufe der Anonymität. Dies resultiert aus dem Stack, dem Routing und der Onion Encryption. Primär sorgt hierbei der Stack durch das Mixen und Verzögern von Nachrichten für einen höheren Schutz gegenüber Tor und Mixnet. Mit Padding und dem optionalen Nutzen von Fake Traffik wird ermöglicht, Daten verschiedener Nutzer aus der perspektive Dritter gleich aussehen zulassen, sowie zu verhindern, dass Dritte sehen können, ob überhaupt kommuniziert wird.
Im Vergleich dazu bietet Tor eine mittlere Anonymität. Es nutzt ebenfalls Onion Encrption. Hinzu kommt das verwenden von Circuit Constructions. Obwohl verschiedene Angriffsmöglichkeiten existieren, erfordert das Durchbrechen der Anonymität den Übergriff auf mehrere Knotenpunkte, mindestens jedoch einem Entry und Exit Node.
Hingegen weist VPN eine geringere Anonymität auf, da sämtliche Daten zentral über einen Server umgeleitet werden. Dieser zentrale Punkt der Datenumleitung erleichtert potenziell die Identifikation von Nutzern im Vergleich zu fortschrittlicheren Anonymisierungstechnologien wie Mixnet und Tor.

\subsubsection{Performance}

VPN ist sehr performant, da Daten nur zweimal zusätzlich verschlüsselt werden (1 mal von Nutzer zu Anbieter, und umgekehrt).
Tor hat eine akzeptaple Performance. Auf der einen Seite verschlüsselt es zwischen allen Knoten das Datenpaket, wodurch es eine höhere Latenz als VPNs hat. Auf der anderen Seite verwendet es keinen Stack, wodurch Narichten verzögert oder gemixt würden, und es nutzt Circuit Constructions, wodurch der Datenverkehr einmal über eine Route festgelegt wird, und danach für etwa 10 Minuten nicht mehr geändert werden muss.
Mixnet ist sehr langsam aufgrund von mehrfachen Verschlüsselungsverfahren bei den Sendern. Es verzögert Nachrichten, mixt diese und wählt verschiede Routen, wodurch es eine hohe Latenz hat, was primär zur niedrigen Performance beiträgt.

\subsection{Fehlerbetrachtung}

Bei einem Qualitative Vergleich treten verschiedene unvermeidbare Fehler auf. Dabei wird unter anderem von modellhaften optimalen Bedingungen ausgegangen. Dabei können jedoch externe Faktoren außer Acht gelassen werden. Dieser Ansatz birgt dadurch das Risiko, dass eben dieser externe Kontext, beispielsweise geographisch bedingt, dafür sorg, dass reale Tests erheblich von den Modellannahmen abweichen. Eine qualitative Untersuchung des Modells kann daher zu erheblichen Diskrepanzen zwischen der Modellvorstellung und den realen Testergebnissen führen, da externe Einflüsse nicht angemessen berücksichtigt werden.

Ein weiterer Aspekt betrifft die Generalisierung und Vereinfachung komplexer Prozesse durch das Modell. Hierbei besteht die Gefahr, dass möglicherweise unbekannte, aber entscheidende Faktoren bei der Modellbildung übersehen oder vernachlässigt werden. Diese Vereinfachung führt zu einer Verfälschung der Ergebnisse, da wichtige Elemente des komplexen Systems unberücksichtigt bleiben\footnote{\cite{DisadvantagesOfQualitativApproaches}, The advantages and disadvantages of using qualitative and quantitative approaches and methods in language “testing and assessment” research: A literature review}.

Darüber hinaus muss für einen Vergleich die zu betrachtenden Objekte, hier anonyme Kommunikationssysteme, nach begrenzt vielen Kriterien untersuchen werden, um begründet ein Urteil zu fällen. Das Begrenzen der Kriterien wird zwangsläufig jedoch zu einer Unvollständigkeit in der Bewertung führen. Die gewählten Kriterien decken möglicherweise nicht alle relevanten Aspekte ab, und die Fokussierung auf nur wenige Kriterien kann zu einem sogenannten \textit{False Balancing} führen. Dies bedeutet, dass ein Modell eine begrenzte Anzahl an Kriterien scheinbar gleich gewichet, obwohl in der abgebildeteten Realität einige Kriterien deutlich schwerer gewichtet sein sollten als andere.