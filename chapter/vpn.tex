\section{VPN}

\subsection{Topologie}

\begin{figure}[h!]
    \centering
    \includesvg{graph/vpn.svg}
    \caption{VPN}
    \label{imgs:vpn}
\end{figure}

Ein virtuelles privates Netzwerk (VPN) ist eine Kommunikationsumgebung, die eine Kombination aus Privatsphäre und Virtualisierung innerhalb einer gemeinsamen Netzwerkinfrastruktur bietet. Um dieses System zu erklären, werden zunächst die Bestandteile des Begriffes einzeln erklärt:

\begin{itemize}

\item Netzwerk: Ein Netzwerk besteht aus Geräten (wie Computern, Druckern, Routern), die mit verschiedenen Methoden kommunizieren können. Diese Kommunikation kann über verschiedene Standorte hinweg erfolgen und wird durch elektronische Signalisierungsspezifikationen und -protokolle ermöglicht.

\item Privat: Im Zusammenhang mit VPNs bedeutet \glqq privat\grqq, dass die Kommunikation zwischen Geräten vertraulich ist. Diejenigen, die nicht an diesem privaten Austausch teilnehmen, haben keine Kenntnis von dessen Inhalt und der privaten Verbindung selbst. Die Gewährleistung des Datenschutzes und der Datensicherheit ist bei der Implementierung von VPNs von entscheidender Bedeutung.

\item Virtuell: \glqq Virtuell\grqq\ bezieht sich auf simulierte oder künstliche Aspekte, die von Computersystemen geschaffen werden, um gemeinsame Ressourcen zu verwalten. Bei VPNs beinhaltet die Virtualisierung die private Kommunikation über eine gemeinsam genutzte Netzinfrastruktur. Das private Netz wird durch logische Partitionierung der gemeinsamen Ressource aufgebaut, nicht durch dedizierte physische Schaltungen.

\item Diskret: VPNs sind diskrete Netzwerke, die getrennt über eine gemeinsam genutzte Infrastruktur arbeiten. Sie bieten exklusive Kommunikationsumgebungen ohne gemeinsame Verbindungspunkte.

\end{itemize}

Kombiniert man diese Aspekte, ist ein VPN ein privates Netzwerk, das durch Virtualisierungsmethoden eingeführt wird. Es handelt sich um eine kontrollierte Kommunikationsumgebung, in der der Zugang nur einer bestimmten Interessengemeinschaft gestattet ist. Diese Umgebung wird durch die Partitionierung eines gemeinsamen Kommunikationsmediums gebildet. Während der Begriff \glqq privates Netzwerk\grqq\ irreführend sein könnte, da alle Netzwerke bis zu einem gewissen Grad auf einer gemeinsamen Infrastruktur beruhen, unterscheiden sich VPNs durch die Segmentierung der Kommunikation innerhalb einer gemeinsamen Infrastruktur.

Ein VPN-Server kann einsehen, was wann von welchem Nutzer an welchen Server gesendet wird. Ein solcher Server wäre in der Lage, sämtliche relavanten Informationen zu erfassen und zu speichern.

In einem formalen Sinne wird ein VPN als eine Kommunikationsumgebung definiert, die den Zugang kontrolliert, um Peer-Verbindungen innerhalb einer bestimmten Interessengemeinschaft zu erleichtern. Diese Umgebung basiert auf der Partitionierung eines gemeinsam genutzten Kommunikationsmediums, das nicht-exklusive Dienste für das Netz bereitstellt\footnote{\cite{DefinitionOfVPN}, What is a VPN?}.

\subsection{Vorteile}

\begin{itemize}
    \item Geoblocking: Anonyme Kommunikationssysteme ermöglichen es den Nutzern, geografische Beschränkungen zu überwinden, die oft durch Geoblocking-Maßnahmen auferlegt werden. Indem sie ihren Standort verschleiern, können Nutzer auf Inhalte und Dienste zugreifen, die normalerweise auf bestimmte Regionen beschränkt sind. Dies ist besonders nützlich für Menschen, die auf reisespezifische oder länderspezifische Inhalte zugreifen möchten, unabhängig von ihrem Aufenthaltsort.
    \item Performance: Topologie erlaubt relativ niedrige Latenz als auch hohen Durchsatz, da Datenverkehr nur über einen Server umgeleitet wird. Im Vergleich zur Performance ohne anonyme Kommunikationssysteme wird somit der Datenverkehr nur einmal zusätzlich verschlüsselt.
    \item VPN ist einfach nachhaltig Monetarisierbar, da VPN Anbieter Nutzern Abo-Modelle oder andere Bezahlmodelle anbieten können. Dies wird dadurch ermöglicht, dass alle Pakete der Nutzer direkt über einen Server des VPN-Anbieters geleitet werden. Der Server kann beim Herstellen einer Verbindung die Anmeldedaten der Nutzer überprüfen und somit feststellen, ob diese Teil für den Service bezahlt haben oder nicht.
\end{itemize}

\subsection{Nachteile}

VPNs bieten Aufgrund ihrer Topologie verschiedene Möglichkeiten für Angreifer. Im Nachfolgenden werden exemplarisch Angriffe auf VPNs betrachtet. Dabei wird in Client-seitige Angriffe und Server-seitige Angriffe unterschieden:

\begin{itemize}
    \item Client-seitige Angriffe: Bei diesen Angriffen werden gefälschte Pakete in den Netzwerkstapel des VPN-Clients eingeschleust. Aufgrund von Network Address Translation (NAT) und Bogon-Filterung wird der Angreifer als netznah zum Client betrachtet, d. h. sie nutzen dasselbe Netz. NAT-Geräte (Network Address Translation) werden in der Regel an Netzgrenzen eingesetzt, um die Adressen von Paketen zu ändern, wenn sie das NAT passieren. NAT ermöglicht es Netzwerken auf beiden Seiten, ihren eigenen Adressraum zu nutzen, Netzwerkadressen zu erhalten und zwischen Adressfamilien wie IPv6 und IPv4 zu übersetzen. Der Hauptzweck von NAT besteht darin, das Problem der Erschöpfung der IPv4-Adressen zu lösen\footnote{\cite{NetworkAddressTranslation}, Network Address Translation: Extending the Internet Address Space}. Bogons sind Präfixe, welche beim Rounting gefiltert werden. Diese werdenn gefiltert, da viele Spam und DDoS Attacken von Bogon-Adressen kommen\footnote{\cite{BogonFiltering}, Bogons and bogon filtering}. Einige Clients, insbesondere solche mit UNIX-ähnlichen Betriebssystemen, die auf dem "weak host model" basieren, unterscheiden Pakete möglicherweise nicht nach der Schnittstelle und ermöglichen es Angreifern, Pakete so zu fälschen, als kämen sie von einem entfernten Anwendungsserver an die IP-Adresse des VPN-Clients innerhalb des VPN-Netzwerks NAT.
    \item Server-seitige Angriffe: Bei diesen Angriffen werden Pakete in den VPN-Server eingeschleust, so dass es so aussieht, als kämen sie von einem entfernten Anwendungsserver. Im Gegensatz zu clientseitigen Angriffen können serverseitige Angriffe von jedem Ort im Internet aus gestartet werden. Es wird davon ausgegangen, dass der Angreifer den verschlüsselten VPN-Verkehr einsehen kann, um serverseitige Angriffe auszuführen, was das Zählen verschlüsselter Pakete oder Bytes im Laufe der Zeit voraussetzt. Die Entschärfung serverseitiger Angriffe gilt als schwieriger, da sich der Angreifer zwischen dem VPN-Server und dem Client befinden muss und die gefälschten Pakete in Bezug auf die Header-Informationen nicht von legitimen Paketen unterschieden werden können\footnote{\cite{AttacksOnVPNs}, Blind In/On-Path Attacks and Applications to VPNs}.
\end{itemize}
