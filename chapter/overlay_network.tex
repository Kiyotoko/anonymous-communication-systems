\section{Overlay Networks}
\label{chap:overlay_networks}

Ein Overlay Network (Abbildung \ref{imgs:overlay_network}) besteht aus Servern, die über das Internet verteilt sind und als zusätzliche Schicht fungieren, die über das bereits bestehende Netzwerk gezogen wird, die einer oder mehreren Anwendungen die Infrastruktur zur Verfügung stellt. Dies bedeutet, dass über die bereits bestehenden physischen Netzwerkverbindungen (Underlay) eine neue Virtuelle Topologie (Overlay) gelegt wird, um mehr Funktionen wie neue Protokolle oder vereinfachtes Routing zu bieten. Entscheidend ist, dass diese Overlays die Verantwortung für die Weiterleitung und Verwaltung von Anwendungsdaten übernehmen und dabei oft von der zugrundelegenden Internet-Infrastruktur abweichen\footnote{\cite{OverlayNetwork}, Overlay Network}.

\begin{figure}[h!]
    \centering
    \includesvg{graph/overlay_network.svg}
    \caption{In einem Overlay Network (Gekennzeichet durch O) wird eine Nachricht von O1 zu O4 gesendet, wodurch eine physische Nachricht im Underlay Network (Gekennzeichnet durch U) von U1 über U2 und U3 zu U4 stattfindet.}
    \label{imgs:overlay_network}
\end{figure}

Overlay Networks ermöglichen eine strukturierte virtuelle Topologie, die über der grundlegenden Transportprotokollebene liegt. Dadurch können sie eine Reihe fortgeschrittener Funktionen anbieten, die über das hinausgehen, was das grundlegende Internet mit seiner physischen Topologie bietet. Sie können Mobilität, individuelles Routing, Quality of Service (QoS), neuartige Adressierung, verbesserte Sicherheit, Multicast und die Verteilung von Inhalten ermöglichen\footnote{\cite{FutureOfTheInternet}, Overlay Networks and the Future of the Internet}.
