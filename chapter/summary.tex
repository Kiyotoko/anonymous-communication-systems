\section{Zusammenfassung}

In dieser Arbeit wurden VPN, Tor und Mixnet qualitativ untersucht und danach miteinander verglichen. Dabei wurde die Topologie betrachtet, sowie technologische Besonderheiten als auch mögliche Angriffe.

Aus den Untersuchungen ergaben sich verschiede Ergbenisse.
VPN bietet höchste Performance, aber geringe Anonymität aufgrund der zentralen Serverstruktur. Es ist geeignet für Streaming und Instant Messaging, wo hohe Performance wichtiger ist als maximale Anonymität.
Mixnet bietet höchste Anonymität, jedoch niedrigere Performance durch ständig wechselnde Routen und Stapelverarbeitung. Es ist geeignet für Banking und Instant Messaging, wo hohe Anonymität wichtiger ist als Performance.
Tor bietet ein ausgewogenes Verhältnis aus Performance und Anonymität. Es ist geeignet für alle drei Anwendungsbereiche aufgrund des ausgewogenen Verhältnisses von Performance und Anonymität.

Die Ergebnisse zeigen, dass kein System gleichzeitig maximale Performance und Anonymität bietet. Welches System für welche Anwendungsbereiche optimal ist, muss somit aus einem Abwägen zwischen Performance und Anonymität für die einzelnen Bereiche bestehen.
