\section{Motivation}

Mit einer fortschreitenden Digitalisierung kommt auch eine verstärkte Vernetzung von Gruppen als auch Firmen. Mit den Vorteilen dieser Entwicklung kommen jedoch auch Schattenseiten. Auf dem Markt der perfekt individuell zugeschnittenen Werbung werden gesammelte Daten ungewollt zur Ware. Eine digitale Vernetzung vereinfacht eine staatliche Überwachung.
Ein Hauptproblem der verstärkten Vernetzung ist eine sinkende Anonymität im Internet.
Ein Lösungsansatz für dieses Problem ist der Einsatz anonymer Kommunikationssysteme wie Mixnet, VPN und Tor. Diese Versprechen die Anonymität ihrer Nutzer zu erhöhen. Doch inwieweit sind sie in der Lage, ihre Nutzer zu anonymisieren? Wie performant sind sie? Und in welchen Bereichen des Alltags ist es sinnvoll, diese einzusetzen?

\section{Fragestellung}

Ziel der Arbeit ist es VPNs, Tor und Mixnets aufgrund ihrer Topologie als auch Funktionsweise zu untersuchen und zu bewerten. Es soll aufgezeigt werden, welches System für welche Bereiche geeignet ist.

\section{Methode}

Die vorliegenden anonymen Kommunikationssysteme sollen zunächst qualitativ untersucht werden. Danach werden diese anhand fest definierter Kriterien bewertet und miteinander verglichen. Abschließend erfolgt eine Zuordnung der Systeme zu möglichen Anwendungsbereichen. Hierbei sollen folgende Kriterien berücksichtigt werden:

\begin{description}
    \item[Anonymität] wird hier definiert, dass ein Nutzer eine Ressource oder Dienst nutzen kann, ohne dass seine Identität offengelegt wird. Sie bezieht sich auf die Fähigkeit eines Kommunikationssystems, die Identität und persönlichen Informationen der Nutzer zu verbergen oder zu verschleiern. Ein anonymes Kommunikationssystem hat zum Ziel, dass Handlungen, Nachrichten oder Interaktionen eines Benutzers keine Rückschlüsse auf seine Identität zulassen. Dies bedeutet auch, dass nicht bekannt ist, welcher Nutzer mit wem kommuniziert\cite{DefinitionOfAnonymity}.

    \item[Performance] bezieht sich auf die Leistungsfähigkeit und Verzögerung in der Übermittlung von Nachrichten eines anonymen Kommunikationssystems. Hierbei sind die Faktoren Latenz und Durchsatz entscheidend. Ein performantes System gewährleistet optimalen Durchsatz als auch Latenz, ohne dabei grundlegende Systemeigenschaften zu beeinträchtigen. Der Durchsatz entspricht der Masse an Daten, die in einer gewissen Zeitspanne übermittelt werden. Der Durchsatz eines Systems soll maximal sein. Die Latenz entspricht der zeitlichen Verzögerung, die zwischen dem Senden einer Nachricht bis zum Empfangen einer Nachricht besteht. Die Latenz eines Systems soll minimal sein\cite{ComputerNetworkPerformanceAnalysis}.
\end{description}

Um diese Kriterien für die einzelnen anonymen Kommunikationssysteme begründet bewerten zu können, wird die Funktionsweise als auch mögliche Angriffe auf ein solches System untersucht. Als Anwendungsbereiche werden \textit{Streaming}, \textit{Messaging} und \textit{Online Banking} betrachtet und definiert.