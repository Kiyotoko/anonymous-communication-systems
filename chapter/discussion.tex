\section{Disskusion}

\subsection{Beantwortung der Leitfrage}

Zunächst werden Typen von Interaktionen realer Anwendungsbereiche im Internet definiert, um anschließend VPN, Tor und Mixnet hinsichtlich Performance und Anonymität zunächst zu bewerten und in die zufor definierten Anwendungsbereiche einzuordnen.

Für die Leitfrage werden die folgenden Bereiche definiert und betrachtet:

Streaming ist die gleichzeitige Übertragung und Wiedergabe von Mediendaten. Es verlangt nach einer hohen Performance, da gleichzeitig Daten übertragen und wiedergegeben werden müssen. Da man selbst kaum private Daten dabei sendet, sondern primär erhält, ist keine hohe Anonymität nötig.

Instant Messaging ist das direkte senden und übertragen von persönlichen Textnachrichten zwischen zwei oder mehr Privatpersonen. Da die Übertragung möglichst unmittelbar ist, sollte eine mittlere Performance gewährleistet sein. Da übertragene Daten jedoch nicht unmittelbar auch wiedergegeben werden und Textnachrichten in der Regel deutlich kleiner als Mediendaten wie Videos oder Bilder sind, wird auch keine so hohe Performance wie beim Streaming gefordert.

Online Banking ist das Abwickeln von Bankgeschäften im Internet. Dabei muss die Sicherheit der Identität als auch der ausgetauchten Daten maximal sein. Daher wird hier eine sehr hohe Anonymität gefordert. Da relativ wenig, dafür jedoch sehr wichtige, Daten ausgetaucht werden, wird nur wenig Performance benötigt.

Die anonymen Kommunikationssysteme werden nun hinsichtlich Anonymität und Performance bewertet.

VPN bietet die höchste Performance und die niedriegste Anonymität. Da über nur einen Server der Datenverkehr geleitet und verschlüsselt wird, ist die Performance sehr hoch. Gleichzeitig führt dies zu einer hohen Zentralität, was eine Schwachstelle für Anonymität und mögliche Angriffe bietet. Es verhindert, dass Services den Ursprung von Anfragen sehen, erlaubt jedoch, dass der VPN-Server sowohl Ursprung als auch Ziel sehen kann.

Tor bietet ein ausgewogenes Verhältnis aus Performance und Anonymität. Es versendet Nachrichten über Entry-, Middle- und Exit-Node, und verschlüsselt über Onion Encryption. Es nutzt Padding, um Nachrichten auf eine gleiche Länge zu bringen. Entry-Nodes können nur den Ursprung sehen, Exit-Nodes nur das Ziel einer Nachricht. Nachrichten werden so versendet, wie sie erhalten wurden. Circuit Construction nutzt für etwa 10 Minuten die selbe Route. Diese beiden Faktoren erlauben es, während dessen durch betrachten des Netzwerkes Nutzer aufgrund ihres Datenverkehrs zu Deanonymisieren.

Mixnet bietet die niedriegste Performance und die höchste Anonymität. Es verwendet eine ähnliche Topologie wie Tor. Es nutzt Onion Encryption und Padding. Zusätzlich wird in Stapeln die Reihenfolge der ankommenden und gesendeten Nachrichten geändert und verzögert. Es nutzt keine Circuit Construction, sondern wählt bei jeder Verbindung eine neu zufällige Route.

\begin{figure}[h!]
    \centering
    \includesvg[width=\linewidth]{graph/systems_discussion.svg}
    \caption{Anonymität und Performance bei VPN, Tor und Mixnet hängen indirekt voneinander ab. Umso höher Performance ist, umso geringer ist die Anonymität, und umgekehrt.}
    \label{imgs:systems_discussion}
\end{figure}

Aus dieser Untersucht zeigt sich, dass hohe Performance und Anonymität sich gegenseitig ausschließen (Abbildung \ref{imgs:systems_discussion}). Dies liegt daran, dass bei den hier untersuchten anonymen Kommunikationssystemen eine höhere Performance niedrigere Anonymität bedeutet, und umgekehrt. Daraus resultiert, dass kein System alleine eine die höchste Performance und Anonymität bietet. Welches System optimal ist, ist ein Abwägen zwischen Performance und Anonymität. Somit haben alle untersuchten Systeme ihre Berechtigung für ihre Anwendungsbeispiele.

VPNs eignen sich für Streaming und Instant Messaging, da hier eine hohe Performance geboten wird und der Mangel an Anonymität vernachlässigbar ist.

Mixnets eignen sich für Banking und Instant Messaging, da hier eine hohe Anonymität geboten wird und der Mangel an Performanc vernachlässigbar ist.

Tor eignet sich aufgrund seiner Ausgewogenheit aus Performance und Anonymität für alle drei Anwendungsbereiche.

\subsection{Reflexion}

Arbeitsprozess? Lernprozess?
