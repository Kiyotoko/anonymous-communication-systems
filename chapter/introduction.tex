\section{Einleitung}

\subsection{Motivation}

Anonyme Kommunikationssysteme sind technologische Strukturen oder Plattformen, die entworfen wurden, um die Identität und persönliche Informationen der Nutzer zu schützen und gleichzeitig sicherzustellen, dass die Kommunikation zwischen den Teilnehmern vertraulich bleibt. Diese Systeme ermöglichen es den Benutzern, Nachrichten auszutauschen, Informationen zu teilen oder an Aktivitäten teilzunehmen, ohne ihre wahre Identität oder Herkunft preiszugeben.

Solche Systeme verwenden verschiedene Techniken, um Anonymität zu gewährleisten, darunter Verschlüsselung, Routing über mehrere Zwischenstationen, Vermischung von Nachrichten und Verwendung von pseudonymen Identitäten. Sie dienen dazu, die Privatsphäre der Nutzer zu schützen, indem sie es Dritten, sei es Regierungen, Unternehmen oder einzelnen Personen, erschweren oder unmöglich machen, die Identität der Kommunikationsteilnehmer zu identifizieren oder die Inhalte ihrer Nachrichten zu überwachen.

Dabei werden diese oft in Umgebungen eingesetzt, in denen der Schutz der Privatsphäre, die Verhinderung von Überwachung oder die Umgehung von Zensur von hoher Bedeutung sind. Anonyme Kommunikationssysteme finden Anwendung in verschiedenen Bereichen, darunter Online-Kommunikation, Whistleblowing, politische Aktivitäten und Forschung, bei denen der Schutz der Identität und der Informationsfreiheit zentral sind. Bekannte Beispiele für solche Systeme sind das Tor-Netzwerk, VPN-Dienste (Virtual Private Networks) und Mixnet-Technologien.

Aufgrund unserer heutigen, immer stärker vernetzten Welt, in der persönliche Daten oft zu Handelsware werden, gewinnt der Schutz der Privatsphäre und Anonymität im Internet unaufhörlich an zunehmender Bedeutung. Diese Entwicklung ist besonders in Ländern wie China von herausragendem Interesse, in denen die staatliche Überwachung eine tiefgreifende Ausprägung erfahren hat. Hierbei erlangen Anonymisierungsverfahren wie VPN, Tor und Mixnet eine herausragende Stellung.

So unter anderem auch Volksrepublik China, welche eine Regierungsform vertritt, die strikte Kontrollen und eine umfassende Zensur des Internets durchgesetzt hat, um nicht nur den Informationsfluss zu steuern, sondern auch die Online-Aktivitäten ihrer Bürger zu überwachen. In diesem spezifischen Kontext haben sich Technologien wie VPN und das Tor als äußerst populäre Anonymisierungsmechanismen etabliert.

Doch in wie weit schützen diese tatsächlich die Daten ihrer Nutzer? Was sind die Stärken und die Schwächen der einzelnen Systeme? 

\subsection{Leitfrage und Ziel}

Die vorliegenden anonymen Kommunikationssysteme sollen qualitativ miteinander verglichen werden, wobei die anschließende Sortierung auf Grundlage verschiedener Kriterien und realen Anwendungsbeispielen erfolgt. Hierbei sollen folgende Kriterien berücksichtigt werden:

\begin{itemize}

\item Anonymität bezieht sich auf die Fähigkeit eines Kommunikationssystems, die Identität und persönlichen Informationen der Nutzer zu verbergen oder zu verschleiern. Ein System mit hoher Anonymität gewährleistet, dass die Handlungen, Nachrichten oder Interaktionen eines Benutzers nicht ohne weiteres auf seine tatsächliche Identität zurückverfolgt werden können. Dieser Schutz der Identität trägt dazu bei, die Privatsphäre der Benutzer zu bewahren und das Risiko der Entdeckung oder Überwachung durch Dritte zu minimieren. Dies bedeutet, dass nicht bekannt ist welche Person mit wem schreibt. Es ist für Dritte nicht möglich, Inhalt und Zeitpunk der Nachricht zu bestimmen, sowie ob überhaupt kommuniziert wird.

\item Dezentralisierung bezieht sich auf die Verteilung von Kontrolle, Daten und Funktionen eines Kommunikationssystems über verschiedene Knotenpunkte oder Teilnehmer hinweg. In einem dezentralisierten System gibt es keine einzige zentrale Entität, die alle Entscheidungen trifft oder alle Daten speichert. Stattdessen teilen sich die Teilnehmer die Verantwortung und tragen zur Erhaltung des Systems bei. Dezentralisierung kann die Widerstandsfähigkeit gegen Störungen erhöhen, die Zensur erschweren und das Vertrauen der Nutzer in das System stärken. Konkret bedeutet dies, dass die Daten über möglichst viele Knotenpunkte gleichmäßig verteilt werden, sodass ein jeder Knotenpunkt die gerings möglichste Masse an Daten erhält.

\item Performance bezieht sich auf die Effizienz und Leistungsfähigkeit eines anonymen Kommunikationssystems. Hierbei werden Faktoren wie Latenzzeit (Verzögerung), Durchsatz (Datenübertragungsrate), Antwortgeschwindigkeit und Kapazität bewertet. Ein performantes System gewährleistet, dass Nachrichten und Daten mit angemessener Geschwindigkeit und Zuverlässigkeit übertragen werden können, ohne dabei die Anonymität oder andere grundlegende Systemeigenschaften grob zu beeinträchtigen. Dies entspricht der Zeit, die ein System benötigt, um eine Nachricht vom Sender zum Empfänger anonym zu versenden.

\item Skalierbarkeit bezieht sich auf die Fähigkeit eines Kommunikationssystems, mit wachsender Nutzerzahl oder steigender Belastung umzugehen, ohne dabei an Effizienz oder Leistungsfähigkeit zu verlieren. Ein skalierbares System kann sowohl bei einer geringen Nutzeranzahl als auch bei einer großen Anzahl von Nutzern reibungslos funktionieren. Skalierbarkeit ist wichtig, um sicherzustellen, dass das System stabil bleibt und eine hohe Qualität der Dienstleistung auch in Zeiten hoher Auslastung gewährleistet ist. Die Qualität Dies bedeutet, dass die Qualität eines Systems mit zunehmens größeren Nutzerzahlen sich nur geringfügig ändert, und im besten Fall sich die Qualität durch den Anstieg der Nutzerzahlen noch erhöht.

\end{itemize}

Die Qualität eines anonymen Kommunikationssystems wird in dieser Arbeit über die Kriterien der Anonymität und Performance definiert. Dezentralisierung und Skalierbarkeit sind hingegen Faktoren, die Gewährleisten, dass die Qualität eines Systems auch in Grenzfällen wie einem externen oder internen Angriff oder technisch bedingten Ausfällen immer noch gewährleistet werden kann.