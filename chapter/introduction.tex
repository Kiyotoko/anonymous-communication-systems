\section{Einleitung}

\subsection{Motivation}

Anonyme Kommunikationssysteme schützen primär ihre Nutzer vor einer Überwachung Dritter. Für den Nutzer bedeutet dies, dass Identität, persönliche Informationen und der Zeitpunk, wann sie mit wem über was schreiben, für andere unbekannt ist. Diese Systeme ermöglichen es den Benutzern, Nachrichten auszutauschen, Informationen zu teilen oder an Aktivitäten teilzunehmen, ohne dass die eigene Identität offengelegt wird.

Solche Systeme verwenden verschiedene Techniken um Anonymität zu gewährleisten, u. a. Verschlüsselung, Routing über mehrere Server oder Vermischung von Nachrichten. Sie dienen dazu, die Privatsphäre der Nutzer zu schützen, indem sie es Dritten, wie Regierungen, Unternehmen oder einzelnen Personen, erschweren oder unmöglich machen, die Identität der Kommunikationsteilnehmer zu identifizieren oder die Inhalte ihrer Nachrichten zu überwachen.
Dabei werden diese oft in Umgebungen eingesetzt, in denen der Schutz der Privatsphäre, die Verhinderung von Überwachung oder die Umgehung von Zensur von hoher Bedeutung sind. Anonyme Kommunikationssysteme finden Anwendung in verschiedenen Bereichen, darunter Online-Kommunikation, Whistleblowing, politische Aktivitäten und Forschung, bei denen der Schutz der Identität zentral sind. Bekannte Beispiele für solche Systeme sind VPN-Dienste, das Tor-Netzwerk und Mixnets.

Aufgrund unserer heutigen, immer stärker vernetzten Welt, in der persönliche Daten oft zu Handelsware werden, gewinnt der Schutz der Privatsphäre und Anonymität im Internet ständig an Bedeutung. Diese Entwicklung ist besonders in Ländern wichtig, in denen staatliche Überwachung ein prägender Bestandteil des Internets ist.
So unter anderem auch China, welche eine Regierungsform vertritt, die strikte Kontrollen und eine umfassende Zensur des Internets durchgesetzt hat, um nicht nur den Informationsfluss zu steuern, sondern auch die Online-Aktivitäten ihrer Bürger zu überwachen. In diesem spezifischen Kontext haben sich dort Technologien wie VPNs und Tor als äußerst populäre Anonymisierungsmechanismen etabliert.

Doch in wie weit schützen diese tatsächlich die Daten ihrer Nutzer? Was sind die Stärken und die Schwächen der einzelnen Systeme?

\subsection{Leitfrage und Ziel}

Ziel der Arbeit ist es VPNs, Tor und Mixnets aufgrund ihrer Topologie und ihrer Funktionsweise zu untersuchen und zu bewerten. Die vorliegenden anonymen Kommunikationssysteme sollen qualitativ miteinander verglichen werden, wobei die anschließende Sortierung auf Grundlage verschiedener Kriterien und realen Anwendungsbereichen erfolgt. Als Anwendungsbereiche werden \textit{Streaming}, \textit{Messaging} und \textit{Online Banking} betrachtet und definiert. Hierbei sollen folgende Kriterien berücksichtigt werden:

\begin{description}
    \item[Anonymität] wird hier definiert, dass ein Nutzer eine Resource oder Dienst nutzen kann, ohne dass seine Identität offengelegt wird. Sie bezieht sich auf die Fähigkeit eines Kommunikationssystems, die Identität und persönlichen Informationen der Nutzer zu verbergen oder zu verschleiern. Ein anonymes Kommunikationssystem hat zum Ziel, dass Handlungen, Nachrichten oder Interaktionen eines Benutzers keine Rückschlüsse auf seine Identität zulassen. Dies bedeutet auch, dass nicht bekannt ist, welcher Nutzer mit wem kommuniziert\cite{DefinitionOfAnonymity}.

    \item[Performance] bezieht sich auf die Leistungsfähigkeit und Verzögerung in der Übermittlung von Nachrichten eines anonymen Kommunikationssystems. Hierbei sind die Faktoren Latenz und Durchsatz entscheidend. Ein performantes System gewährleistet optimalen Durchsatz als auch Latenz, ohne dabei grundlegende Systemeigenschaften zu beeinträchtigen. Der Durchsatz entspricht der Masse an Daten, die in einer gewissen Zeitspanne übermittelt werden. Der Durchsatz eines Systems soll maximal sein. Die Latenz entspricht der zeitlichen Verzögerung, die zwischen dem Senden einer Nachricht bis zum Empfangen einer Nachricht besteht. Die Latenz eines Systems soll minimal sein\cite{ComputerNetworkPerformanceAnalysis}.
\end{description}
