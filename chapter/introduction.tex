\section{Einleitung}

\subsection{Motivation}

Anonyme Kommunikationssysteme schützen primär ihre Nutzer vor einer Überwachung Dritter. Dies bedeutet für die Nutzer, dass Identität, persönliche Informationen, wann sie mit wem über was schreiben zu jedem Zeitpunk für andere unbekannt ist. Diese Systeme ermöglichen es den Benutzern, Nachrichten auszutauschen, Informationen zu teilen oder an Aktivitäten teilzunehmen, ohne selbst Informationen von sich preisgeben zu müssen.

Solche Systeme verwenden verschiedene Techniken um Anonymität zu gewährleisten, darunter Verschlüsselung, Routing über mehrere Zwischenstationen, Vermischung von Nachrichten und Verwendung von pseudonymen Identitäten. Sie dienen dazu, die Privatsphäre der Nutzer zu schützen, indem sie es Dritten, sei es Regierungen, Unternehmen oder einzelnen Personen, erschweren oder unmöglich machen, die Identität der Kommunikationsteilnehmer zu identifizieren oder die Inhalte ihrer Nachrichten zu überwachen.

Dabei werden diese oft in Umgebungen eingesetzt, in denen der Schutz der Privatsphäre, die Verhinderung von Überwachung oder die Umgehung von Zensur von hoher Bedeutung sind. Anonyme Kommunikationssysteme finden Anwendung in verschiedenen Bereichen, darunter Online-Kommunikation, Whistleblowing, politische Aktivitäten und Forschung, bei denen der Schutz der Identität und zentral sind. Bekannte Beispiele für solche Systeme sind das VPN-Dienste, Tor-Netzwerk und Mixnets.

Aufgrund unserer heutigen, immer stärker vernetzten Welt, in der persönliche Daten oft zu Handelsware werden, gewinnt der Schutz der Privatsphäre und Anonymität im Internet ständig an Bedeutung. Diese Entwicklung ist besonders in Ländern wichtig, in denen staatliche Überwachung ein prägender Bestandteil des Internets ist.

So unter anderem auch Volksrepublik China, welche eine Regierungsform vertritt, die strikte Kontrollen und eine umfassende Zensur des Internets durchgesetzt hat, um nicht nur den Informationsfluss zu steuern, sondern auch die Online-Aktivitäten ihrer Bürger zu überwachen. In diesem spezifischen Kontext haben sich Technologien wie VPN und Tor als äußerst populäre Anonymisierungsmechanismen etabliert.

Doch in wie weit schützen diese tatsächlich die Daten ihrer Nutzer? Was sind die Stärken und die Schwächen der einzelnen Systeme? 

\subsection{Leitfrage und Ziel}

Ziel der Arbeits ist es VPNs, Tor und Mixnets aufgrund ihrer Topologie zu untersuchen und zu bewerten. Die vorliegenden anonymen Kommunikationssysteme sollen qualitativ miteinander verglichen werden, wobei die anschließende Sortierung auf Grundlage verschiedener Kriterien und realen Anwendungsbereichen erfolgt. Hierbei sollen folgende Kriterien berücksichtigt werden:

\begin{itemize}
    \item \textbf{Anonymität} ist damit definiert, dass ein Nutzer eine Resource oder Dienst nutzen kann, ohne seine Identität preiszugeben. Es bezieht sich auf die Fähigkeit eines Kommunikationssystems, die Identität und persönlichen Informationen der Nutzer zu verbergen oder zu verschleiern. Ein System gewährleistet, dass Handlungen, Nachrichten oder Interaktionen eines Benutzers keine Rückschlüsse auf seine Identität zulassen. Dieser Schutz der Identität trägt dazu bei, die Privatsphäre der Benutzer zu bewahren und das Risiko der Entdeckung oder Überwachung durch Dritte zu minimieren. Dies bedeutet, dass nicht bekannt ist welcher Nutzer mit wem kommuniziert\footnote{\cite{DefinitionOfAnonymity}, Privacy and Anonymity}.

    \item \textbf{Performance} bezieht sich auf die Effizienz, Leistungsfähigkeit und Verzögerung in der Übermittlung von Nachrichten eines anonymen Kommunikationssystems. Hierbei werden die Faktoren Latenz und Durchsatz gewertet. Ein performantes System gewährleistet durchgängig einen stabilen Durchsatz als auch Latenzzeit, ohne dabei grundlegende Systemeigenschaften zu beeinträchtigen. Der Durchsatz entspricht der Masse an Daten, die in einer gewissen Zeitspanne übermittelt werden. Der Durchsatz eines Systems soll maximal sein. Die Latenz entspricht der zeitlichen Verzögerung, die zwischen dem Senden einer Nachricht bis zum Empfangen einer Nachricht besteht. Die Latenz eines Systems soll minimal sein\footnote{\cite{ComputerNetworkPerformanceAnalysis}, Real-time computer network performance analysis based on ISO/OSI transport service definition}.
\end{itemize}
