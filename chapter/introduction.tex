\section{Einleitung}

\subsection{Motivation}

In einer zunehmend vernetzten Welt, in der persönliche Daten oft zur Ware werden, gewinnt der Schutz der Privatsphäre und Anonymität im Internet immer mehr an Bedeutung. Insbesondere in Ländern wie China, in denen die staatliche Überwachung stark ausgeprägt ist, sind Anonymisierungsverfahren wie VPN, Tor und Mixnet von großer Bedeutung.

China steht für eine Regierung, die strenge Internetkontrollen und -zensur implementiert hat, um Informationen zu kontrollieren und die Aktivitäten ihrer Bürger online zu überwachen. In diesem Kontext sind VPN (Virtual Private Network) und Tor (The Onion Router) zu den beliebtesten Anonymisierungsverfahren geworden.

\subsection{Leitfrage und Ziel}

Die anonymen Kommunikationssystem sollen anhand der Folgenden Kriterien analysiert und aufgrund ihres Fazits sortiert werden. Das erste Kriterium ist Anonymität. Wie gut ist die Identität und Privatsphäre des Benutzers geschützt? Das nächste Kriterium ist Dezentralisierung. Hier soll festgestellt werden, wie zentralisiert und dezentralisiert das System ist. Ein weiteres Kriterium ist Skalierbarkeit.