\section{Einleitung}

\subsection{Motivation}

In unserer heutigen, immer stärker vernetzten Welt, in der persönliche Daten oft zu Handelsware werden, gewinnt der Schutz der Privatsphäre und Anonymität im Internet unaufhörlich an zunehmender Bedeutung. Diese Entwicklung ist besonders in Ländern wie China von herausragendem Interesse, in denen die staatliche Überwachung eine tiefgreifende Ausprägung erfahren hat. Hierbei erlangen Anonymisierungsverfahren wie Virtual Private Networks (VPN), das Tor-Netzwerk und Mixnet eine herausragende Stellung.

Die Volksrepublik China vertritt eine Regierungsform, die strikte Kontrollen und eine umfassende Zensur des Internets durchgesetzt hat, um nicht nur den Informationsfluss zu steuern, sondern auch die Online-Aktivitäten ihrer Bürger zu überwachen. In diesem spezifischen Kontext haben sich Technologien wie Virtual Private Networks (VPN) und das Tor-Netzwerk (The Onion Router) als äußerst populäre Anonymisierungsmechanismen etabliert.

\subsection{Leitfrage und Ziel}

Die vorliegenden anonymen Kommunikationssysteme sollen einer gründlichen Analyse unterzogen werden, wobei die anschließende Sortierung auf Grundlage verschiedener Kriterien erfolgt. Hierbei sollen folgende Kriterien berücksichtigt werden:

\begin{itemize}

\item Anonymität bezieht sich auf die Fähigkeit eines Kommunikationssystems, die Identität und persönlichen Informationen der Nutzer zu verbergen oder zu verschleiern. Ein System mit hoher Anonymität gewährleistet, dass die Handlungen, Nachrichten oder Interaktionen eines Benutzers nicht ohne weiteres auf seine tatsächliche Identität zurückverfolgt werden können. Dieser Schutz der Identität trägt dazu bei, die Privatsphäre der Benutzer zu bewahren und das Risiko der Entdeckung oder Überwachung durch Dritte zu minimieren.

\item Dezentralisierung bezieht sich auf die Verteilung von Kontrolle, Daten und Funktionen eines Kommunikationssystems über verschiedene Knotenpunkte oder Teilnehmer hinweg. In einem dezentralisierten System gibt es keine einzige zentrale Entität, die alle Entscheidungen trifft oder alle Daten speichert. Stattdessen teilen sich die Teilnehmer die Verantwortung und tragen zur Erhaltung des Systems bei. Dezentralisierung kann die Widerstandsfähigkeit gegen Störungen erhöhen, die Zensur erschweren und das Vertrauen der Nutzer in das System stärken.

\item Performance bezieht sich auf die Effizienz und Leistungsfähigkeit eines anonymen Kommunikationssystems. Hierbei werden Faktoren wie Latenzzeit (Verzögerung), Durchsatz (Datenübertragungsrate), Antwortgeschwindigkeit und Kapazität bewertet. Ein performantes System gewährleistet, dass Nachrichten und Daten mit angemessener Geschwindigkeit und Zuverlässigkeit übertragen werden können, ohne dabei die Anonymität oder andere grundlegende Systemeigenschaften zu beeinträchtigen.

\item Skalierbarkeit bezieht sich auf die Fähigkeit eines Kommunikationssystems, mit wachsender Nutzerzahl oder steigender Belastung umzugehen, ohne dabei an Effizienz oder Leistungsfähigkeit zu verlieren. Ein skalierbares System kann sowohl bei einer geringen Nutzeranzahl als auch bei einer großen Anzahl von Nutzern reibungslos funktionieren. Skalierbarkeit ist wichtig, um sicherzustellen, dass das System stabil bleibt und eine hohe Qualität der Dienstleistung auch in Zeiten hoher Auslastung gewährleistet ist.

\end{itemize}