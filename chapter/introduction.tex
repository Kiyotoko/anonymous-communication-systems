\section{Einleitung}

\subsection{Motivation}

Anonyme Kommunikationssysteme sind Systeme die primär ihre Nutzer vor einer Überwachung Dritter schützen. Dies bedeutet kongret für die Nutzer, dass die Identität, persönliche Informationen, wann sie mit wem über was schreiben zu jedem Zeitpunk für andere nicht einsehbar ist. Diese Systeme ermöglichen es den Benutzern, Nachrichten auszutauschen, Informationen zu teilen oder an Aktivitäten teilzunehmen, ohne selbst Informationen von sich preisgeben zu müssen.

Solche Systeme verwenden verschiedene Techniken, um Anonymität zu gewährleisten, darunter Verschlüsselung, Routing über mehrere Zwischenstationen, Vermischung von Nachrichten und Verwendung von pseudonymen Identitäten. Sie dienen dazu, die Privatsphäre der Nutzer zu schützen, indem sie es Dritten, sei es Regierungen, Unternehmen oder einzelnen Personen, erschweren oder unmöglich machen, die Identität der Kommunikationsteilnehmer zu identifizieren oder die Inhalte ihrer Nachrichten zu überwachen.

Dabei werden diese oft in Umgebungen eingesetzt, in denen der Schutz der Privatsphäre, die Verhinderung von Überwachung oder die Umgehung von Zensur von hoher Bedeutung sind. Anonyme Kommunikationssysteme finden Anwendung in verschiedenen Bereichen, darunter Online-Kommunikation, Whistleblowing, politische Aktivitäten und Forschung, bei denen der Schutz der Identität und der Informationsfreiheit zentral sind. Bekannte Beispiele für solche Systeme sind das Tor-Netzwerk, VPN-Dienste (Virtual Private Networks) und Mixnet-Technologien.

Aufgrund unserer heutigen, immer stärker vernetzten Welt, in der persönliche Daten oft zu Handelsware werden, gewinnt der Schutz der Privatsphäre und Anonymität im Internet unaufhörlich an zunehmender Bedeutung. Diese Entwicklung ist besonders in Ländern wie China von herausragendem Interesse, in denen die staatliche Überwachung eine tiefgreifende Ausprägung erfahren hat.

So unter anderem auch Volksrepublik China, welche eine Regierungsform vertritt, die strikte Kontrollen und eine umfassende Zensur des Internets durchgesetzt hat, um nicht nur den Informationsfluss zu steuern, sondern auch die Online-Aktivitäten ihrer Bürger zu überwachen. In diesem spezifischen Kontext haben sich Technologien wie VPN und das Tor als äußerst populäre Anonymisierungsmechanismen etabliert.

Doch in wie weit schützen diese tatsächlich die Daten ihrer Nutzer? Was sind die Stärken und die Schwächen der einzelnen Systeme? 

\subsection{Leitfrage und Ziel}

Die vorliegenden anonymen Kommunikationssysteme sollen qualitativ miteinander verglichen werden, wobei die anschließende Sortierung auf Grundlage verschiedener Kriterien und realen Anwendungsbeispielen erfolgt. Hierbei sollen folgende Kriterien berücksichtigt werden:

\begin{itemize}

\item Anonymität bezieht sich auf die Fähigkeit eines Kommunikationssystems, die Identität und persönlichen Informationen der Nutzer zu verbergen oder zu verschleiern. Ein System mit hoher Anonymität gewährleistet, dass die Handlungen, Nachrichten oder Interaktionen eines Benutzers nicht zurückverfolgt werden und mitgelesen werden können. Dieser Schutz der Identität trägt dazu bei, die Privatsphäre der Benutzer zu bewahren und das Risiko der Entdeckung oder Überwachung durch Dritte zu minimieren. Dies bedeutet, dass nicht bekannt ist welcher Nutzer mit wem schreibt.

\item Performance bezieht sich auf die Effizienz, Leistungsfähigkeit und Verzögerung in der Übermittlung von Nachrichten eines anonymen Kommunikationssystems. Hierbei werden Faktoren wie Latenzzeit (Verzögerung), Durchsatz (Datenübertragungsrate) und Kapazität gewertet. Ein performantes System gewährleistet durchgängig einen stabilen Durchsatz als auch Latenzzeit, ohne dabei die Anonymität oder andere grundlegende Systemeigenschaften grob zu beeinträchtigen. Der Durchsatz entspricht der Masse an Daten, die in einer gewissen Zeitspanne übermittelt werden. Die Latenzzeit entspricht der Zeit, die ein System benötigt, um eine Nachricht vom Sender zum Empfänger zu versenden\footnote{\cite{ComputerNetworkPerformanceAnalysis}, Real-time computer network performance analysis based on ISO/OSI transport service definition}.

\item Dezentralisierung ist die Verteilung von Anfragen und Daten innerhalb eines Netzwerkes. In einem maximal zentralisiertem Netzwerk erhält ein Knoten alle Anfragen und Daten von allen anderen Knoten. Dies führt dazu, dass ein Knoten allein unter einer hohen Auslastung steht und Zugriff auf alle Daten hat. Dezentralisierung ist hierbei ein Mittel, um hohe Anonymität und Performance zu erreichen, da in einem dezentralisierten Netzwerk sich die Anfragen und Daten über alle Knoten gleichmäßig verteilen, was die Anfragen pro Knoten senkt und die Performance erhöht. Die Anonymität wird dadurch unter der Bedingung erhöht, dass ein Dritter alle verstreuten Daten braucht, um einen Nutzer zu deanonymisieren.

\item Skalierbarkeit ist die Möglichkeit innerhalb eines Systems, dieses einfach zu erweitern oder zu verändern. Dies bedeutet, dasseEin System mit einer hohen Skalierbarkeit erlaubt, einfach neue Knoten und Funktionen hinzuzufügen oder alte zu löschen. Skalierbarkeit ist kein Kriterium, welches Anonymität oder Performance direkt oder indirekt verbessert, sondern vielmehr eine garantie, dass diese auch in Zukunft noch besteht bezieht weise erhöht werden kann\footnote{\cite{ScalabilityOfNetworking}, On scalability of software-defined networking}.

\end{itemize}

Die Qualität eines anonymen Kommunikationssystems wird in dieser Arbeit über die Kriterien der Anonymität und Performance definiert. Dezentralisierung ist hingegen ein Mittel, um Anonymität und Performance zu erreichen. Skalierbarkeit ist dabei ein Faktor, der Gewährleistet, dass die Qualität eines Systems auch langfristig noch in Zukunft immer noch besteht und gewarted werden kann.